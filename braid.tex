\documentclass[a4paper,12pt]{article}

\usepackage[english,greek]{babel}
\usepackage[utf8x]{inputenc}
\usepackage{kerkis}
\usepackage{natbib}
\usepackage{amssymb}
\usepackage{amsthm}
\usepackage{amsmath}
% \usepackage{algorithm}
% \usepackage{algpseudocode}
\usepackage{listings}
\usepackage{textcomp}
\usepackage[margin=0.7in]{geometry}
\usepackage{multirow}
\usepackage{multicol}
\usepackage{indentfirst}

\usepackage{tikz}
\usetikzlibrary{shapes,arrows}

% \algsetup{indent=2em}
% \newcommand{\factorial}{\ensuremath{\mbox{\sc Factorial}}}

  
\title{
Αλγόριθμος \textlatin{Braid}: Εξόρυξη ροών δεδομένων μέσω ομαδοποίησης λόγω συσχέτισης με καθυστέρηση
}
\author{Πυργερής Γιώργος\\
\begin{small}\textlatin{\texttt{pyrgeris@ceid.upatras.gr}}\end{small}}


\begin{document}

\maketitle

\section*{Περίληψη}

\par Η εργασία έχει ως σκοπό την παρακολούθηση πολλαπλών αριθμητικών ροών δεδομένων και τον καθορισμό των ζευγών που συσχετίζονται με κάποια καθυστέρηση, καθώς και ο υπολογισμός της τιμής της καθυστέρησης αυτής.
\par Προτείνεται η χρήση του αλγόριθμου \textlatin{Braid} . Πρόκειται για μία μέθοδο υπολογισμού της συσχέτισης με καθυστέρηση (\textlatin{lag correlation}), μεταξύ των ροών δεδομένων. Υποστηρίζει ροές δεδομένων με σχεδόν άπειρο μήκος, 
κλιμακωτά, γρήγορα και με μικρό υπολογιστικό κόστος .
\par Επίσης παρουσιάζεται μία θεωρητική ανάλυση βασισμένη στο θεώρημα \textlatin{Nyquist} σύμφωνα με την οποία, αποδεικνύεται ότι ο \textlatin{Braid}
υπολογίζει το \textlatin{lag correlation} με μικρό και συνήθως μηδενικό σφάλμα.
Το μέγιστο σχετικό σφάλμα υπολογίζεται γύρω στο 1\%. Η ταχύτητα επεξεργασίας του αλγόριθμου είναι γύρω στις 14.000 φορές πιο γρήγορη από την απλή υλοποίηση που θα μπορούσε κάποιος να εφαρμόσει για την επίλυση του προβλήματος.

\section*{Εισαγωγή}

\par \underline{Αρχικό Πρόβλημα}: Δοθέντων δύο ακολουθιών που εξελίσσονται παράλληλα στο χρόνο, ίδιου μήκους $n$ , πρέπει να μπορεί να υπολογιστεί ανά πάσα χρονική στιγμή α) αν υπάρχει συσχέτιση μέσω καθυστέρησης (\textlatin{lag correlation}), τιμής $l$, μεταξύ τους και β) αν ναι, ποια είναι η τιμή της καθυστέρησης αυτής.
\par \underline{Κύριο Πρόβλημα}: Δοθέντων $k$ ακολουθιών που εξελίσσονται παράλληλα στο χρόνο, ίδιου μήκους $n$, πρέπει να μπορεί να υπολογιστεί ανά πάσα χρονική στιγμή α) ποιά ζεύγη έχουν συσχέτιση μέσω καθυστέρησης (\textlatin{lag correlation}) τιμής $l$, μεταξύ τους και β) να μπορεί να επιστρέφεται τιμή της καθυστέρησης αυτής.
\par Διαισθητικά δύο ακολουθίες δεδομένων έχουν συσχέτιση με καθυστέρηση (\textlatin{lag correlation}) τιμής $l$, αν δείχουν σχεδόν ίδιες αν η μία καθυστερήσει κατά $l$ χρονικές στιγμές. Αν οι δύο ακολουθίες ήταν στατικές το πρόβλημα θα ήταν τετριμένο. Απλά θα έπρεπε να υπολογιστεί ο συντελεστής συσχέτισης(\textlatin{correlation factor $R(l)$}) μέσω της συνάρτησης \textlatin{CCF (cross-correlation function)} και να επιστραφεί η τιμή της καθυστέρησης $l$, τη στιγμή που μεγιστοποιείται ο συντελεστής. Αλλά οι δυο ακολουθίες δεδομένων $X,Y$ συνεχώς μεγαλώνουν σε βάθος χρόνου. Χρειαζόμαστε μία μέθοδο με τα εξής χαρακτηριστικά :
\begin{enumerate}
 \item Επεξεργασία ανα πάσα στιγμή και ταχύτατα. Ο χρόνος επεξεργασίας θα πρέπει να είναι υπό-γραμμικός (στη βέλτιστη περίπτωση ακαριαίος) σε ακολουθίες με μήκος $n$.
 \item Eυελιξία. Οι απαιτήσεις μνήμης πρέπει να είναι υπό-γραμμικές σε σχέση με το μήκος $n$ των ακολουθιών.
 \item Ακρίβεια. Δεδομένου ότι τα ακριβή αποτελέσματα απαιτούν πάρα πολύ χώρο και χρόνο, χρειαζόμαστε προσεγγίσεις. Τέτοιου είδους προσέγγιση εισάγει ένα μικρό σφάλμα.
\end{enumerate}
\noindent Ο αλγόριθμος \textlatin{Braid} είναι ο πρώτος αλγόριθμος που καλύπτει όλα τα παραπάνω χαρακτηριστικά.

\section*{Προτεινόμενη Μέθοδος}

\par Μια ροή δεδομένων \textlatin{X} είναι μία διακριτή ακολουθία αριθμών \{$x_1, x_2 , ... , x_n$\} όπου $x_n$ είναι η πιο πρόσφατη τιμή. Το μέγεθος $n$ αυξάνεται σε κάθε χρονική στιγμή. Ο ορισμός του συντελεστή συσχέτισης $R(0)$ μεταξύ δύο χρονικών ακολουθιών $X$ και $Y$, ίδιου μήκους $n$ και μηδενικής καθυστέρησης $l=0$ είναι ευρέως γνωστός σαν συντελεσής του \textlatin{Pearson($p$)}. 

 \begin{equation*}
  p = R(0) = \frac{\sum_{t}(x_t - \bar{x})*(y_t - \bar{y})}{\sigma(x)*\sigma(y)}
 \end{equation*}
 
\noindent όπου $\bar{x}$, $\bar{y}$ είναι οι μέσοι όροι των ακολουθιών $X$ και $Y$. Για τιμή $l\geq 0$ όπου $l$, η καθυστέρηση, το $R(l)$ μας δίνει το συντελεστή συσχέτισης, όταν η ακολουθία $X$ είναι καθυστερημένη κατά $l$. Το $R(l)$ δίνεται από τον τύπο :

 \begin{equation*}
  R(L) = \frac{\sum_{t=l+1}^{n}(x_t - \bar{x})*(y_{t-l} - \bar{y})}{\sqrt{\sum_{t=l+1}^{n}(x_t - \bar{x})^2)}*\sqrt{\sum_{t=1}^{n-l}(y_t - \bar{y})^2)}} ,\;\; \bar{x}=\frac{1}{n-l}\sum\limits_{t=l+1}^{n}x_t ,\;\; \bar{y}=\frac{1}{n-l}\sum\limits_{t=1}^{n-l}y_t
 \end{equation*}

\noindent Τονίζεται ότι σημαντικές είναι μόνο οι απόλυτες τιμές του $R(l)$.
 \begin{equation*}
  score(l) = \mid R(l) \mid
 \end{equation*}
 
 \noindent \textbf{\underline{ΟΡΙΣΜΟΣ - \textlatin{Lag Correlation}:}} Δύο ακολουθίες $X$, $Y$ έχουν συσχέτιση με καθυστέρηση(\textlatin{Lag Correlation}) με τιμή $l$, και συγκεκριμένα το $X$ καθυστερεί το $Y$ κατά $l$, όταν :
 
 \begin{enumerate}
 \item[1.] Η απόλυτη τιμή του $R(l)$ μεταξύ του $x_t$ και του $y_{t-l}$ είναι πάνω από ένα φράγμα $\gamma$, έστω $\gamma=0.4$ και η τιμή αυτή είναι τοπικό μέγιστο.
 \item[2.] Και αυτό είναι το πρώτο τέτοιο μέγιστο, αν υπάρχουν και άλλα τοπικά μέγιστα.
 \end{enumerate}
 
 \noindent Υπάρχει περιορισμός για τη μέγιστη τιμή του $l$ και καθορίζεται από τον τύπο $m=\frac{n}{2}$, δηλαδή η μέγιστη τιμή της καθυστέρησης πρέπει να έιναι ίση ή μικρότερη από το μισό της ακολουθίας.

\end{document}